\documentclass[9pt]{extarticle}

\usepackage{extsizes}
\usepackage{hyperref}
\usepackage[margin=0.5in]{geometry}
\usepackage{paralist}

\setcounter{secnumdepth}{0}
\setlength{\parindent}{0pt}
\setlength{\parskip}{0pt}
\setlength{\pltopsep}{1pt}
\pagenumbering{gobble}

\newcommand{\myline}{\rule[\baselineskip]{\linewidth}{1pt}}

\begin{document}
\begin{center}
\Huge
\textbf{Aaron Trowbridge}\\

\normalsize
(610) 955-1580 $\cdot$ \href{mailto:aaron.j.trowbridge@gmail.com}{aaron.j.trowbridge@gmail.com} $\cdot$ \href{https://aarontrowbridge.github.io/}{aarontrowbridge.github.io} \\

\end{center}

% \Large{\textbf{Education}}

\section{Education}

\myline


\large\textbf{Syracuse University} 
\vspace{7pt}

\normalsize
\begin{compactitem}
\setlength\itemsep{0em}
\item B.S. in Physics, with distinction (3.6 GPA); B.S. in Mathematics (3.8 GPA) \hfill \small Sep 2015 -- Dec 2020 
% \item \normalsize One Semester as Graduate Physics Student and TA (4.0 GPA) [atypical situation due to COVID] \hfill \small Jan 2021 -- May 2021

\end{compactitem}


\section{Experience}

\myline

% \large\textbf{Private Tutor} \hfill \small Mar 2021 -- May 2022 
% \normalsize

% \begin{compactitem}
% \item Taught subjects including: physics, statistics, calculus, and programming in python
% \end{compactitem}


\vspace{7pt}
\large\textbf{Research Associate} \normalsize (Carnegie Mellon Robotics Exploration Lab) \hfill \small Aug 2022 -- Present 
\normalsize


\begin{compactitem}
\item Researching quantum optimal control under Prof. Zac Manchester and Prof. David Schuster.  
% \item Gave a talk titled ``Quantum Collocation and Iterative Learning Control" at the SIAM CSE 2023 \\ conference in Amsterdam.
\item Developed and tested a novel pulse generation method on hardware systems. 
% with a paper on ArXiv: \\ \href{https://arxiv.org/abs/2305.03261}{\underline{Direct Collocation for Quantum Optimal Control}}. 
\item Developed the following open source software packages: \\ \href{https://github.com/aarontrowbridge/QuantumCollocation.jl}{QuantumCollocation.jl}, \href{https://github.com/aarontrowbridge/IterativeLearningControl.jl}{IterativeLearningControl.jl}, and \href{https://github.com/aarontrowbridge/NamedTrajectories.jl}{NamedTrajectories.jl}. 
\normalsize
\end{compactitem}

% \vspace{7pt}
% \large\textbf{Research Assistant} \normalsize (Syracuse University Physics Department)

% \begin{compactitem}
% \normalsize
% \item Lattice quantum gravity group, under Prof. Jack Laiho \hfill \small May 2020 -- May 2021 \normalsize
% \item \href{https://bplourde.expressions.syr.edu/}{\underline{Plourde Research Lab}} (superconducting quantum devices), under Prof. Britton Plourde \hfill \small May 2018 -- Dec 2020
% \end{compactitem}



\vspace{7pt}
\large\textbf{Data Engineering Intern} \normalsize (CatalystIQ) \hfill \small May 2022 -- Aug 2022 

\begin{compactitem}
  \normalsize
\item Developed backend components for an automated content tagging platform used in marketing analytics tasks.
\item Implemented data ingestion pipelines for large continuously updating healthcare datasets utilizing \\ AWS services combined with Snowflake databases. 
\end{compactitem}

\vspace{7pt}
\large\textbf{Teaching Assistant} \normalsize (Syracuse University Physics Department)

\begin{compactitem}
\normalsize
\item One semester as graduate TA: \href{https://walterfreeman.github.io/phy211/}{PHY 211} taught by Prof. Walter Freeman \hfill \small Jan 2021 -- May 2021 \normalsize
\item Four semesters as undergrad TA: astronomy, mechanics, E $\&$ M, computational physics  \hfill \small Jan 2019 -- Dec 2020 \normalsize
\end{compactitem}

\section{Talks \& Publications} 
\myline

\textbf{Quantum Collocation and Iterative Learning Control} \hfill \small \textit{Talk, SIAM CSE23}, March 2023 \normalsize
\begin{compactitem}
  \item Speaker: \textbf{Aaron Trowbridge} 
\end{compactitem}

\textbf{Piccolo.jl: An integrated quantum optimal control stack} \hfill \small \textit{Talk, JuliaCon 2023, \href{https://www.youtube.com/watch?v=NBdck6UX0Tc}{YouTube}}, July 2023 \normalsize
\begin{compactitem}
  \item Speaker: \textbf{Aaron Trowbridge} and Aditya Bhardwaj 
\end{compactitem}



\textbf{Direct Collocation for Quantum Optimal Control}  \hfill \small \textit{Paper and Talk, IEEE QCE23 (2nd best paper award), \href{https://arxiv.org/abs/2305.03261}{\underline{ArXiv}}}, Sept. 2023 \normalsize
\begin{compactitem}
  \item Authors: \textbf{Aaron Trowbridge}, Aditya Bhardwaj, Kevin He, David I. Schuster, and Zachary Manchester 
\end{compactitem}







\section{Projects}

\myline

% \large\textbf{Personal Website and Blog}
% \normalsize

% \begin{compactitem}
% \item Used the Hugo static website framework to develop my personal website and blog; learning HTML, CSS, Markdown, Go, and (just a little) JavaScript in the process. 
% \item I have written \href{https://aarontrowbridge.github.io/posts/}{\underline{blog posts}} about interactive web-based plotting with Julia, quantum field theory, and quantitative finance, as well as other topics, and plan on writing more about data science and machine learning in the near future. 
% \end{compactitem}


\vspace{2.5pt}
\large\textbf{Superconducting Quantum Devices}
\normalsize
\begin{compactitem}

\item Extracted device parameters from spectroscopic data using Python and built simulations of Josephson Junction circuit \\ dynamics in Julia advised by Prof. Britton Plourde. 

\item Simulation code can be found \href{https://github.com/aarontrowbridge/cQED}{\underline{here}}.

\end{compactitem}

\vspace{2.5pt}
\large\textbf{Quantum Computation}
\normalsize

\begin{compactitem}
\item Implemented a custom quantum gate programming language and virtual quantum processor, in Julia. 

\item Code can be found \href{https://github.com/aarontrowbridge/QuIPS}{\underline{here}}.

\end{compactitem}

\vspace{2.5pt}
\large\textbf{Monte Carlo Methods for Lattice Quantum Gravity}
\normalsize

\begin{compactitem}
\item Developed a novel rejection-free variant of the Metropolis algorithm specially designed for dynamical triangulation simulations of quantum gravity, advised by Prof. Jack Laiho and Prof. Walter Freeman.
\item A recorded talk I gave can be found on \href{https://www.youtube.com/watch?v=_Ppx0e3aG-E&t=2s}{\underline{youtube}}, a short blog post can be found \href{https://aarontrowbridge.github.io/posts/the-freeman-method/}{\underline{here}}, and a GitHub repo \href{https://github.com/aarontrowbridge/Ising}{\underline{here}}.
\end{compactitem}



\vspace{2.5pt}
\large\textbf{Deep Generative Models}
\normalsize

\begin{compactitem}
\item Implemented generative adversarial networks (GANs) for image generation from scratch in Julia using Flux.jl. 
\item Conducted additional research on conditional GANs and various types of variational autoencoders (VAEs).
\item Code can be found \href{https://github.com/aarontrowbridge/FluxGAN.jl}{\underline{here}} and a blog post \href{https://aarontrowbridge.github.io/posts/generative-adversarial-nets/}{\underline{here}}.
% \item Implemented various variational autoencoder models in python using the PyTorch library. 
% \item Read various papers about optimized generative techniques, including $\beta$-VAE, Wasserstein GAN/VAE, and Riemannian Manifold Hamiltonian Monte Carlo sampling. 
% \item Currently researching other generative techniques, e.g. flow based models using gauge equivariant layers to sample configurations for lattice gauge theory (a technique for simulating quantum field theories).
\end{compactitem}



\section{Additional Information}
\myline

\normalsize

\begin{tabular}{ll}
\textbf{\textit{Programming}}: & Julia, Python, SQL, AWS, Git, \LaTeX \\
% \textbf{\textit{Operating Systems}}: & Linux (Arch, Manjaro i3), MacOS, Windows 10, Arduino, Visual Studio Code \\ 
\textbf{\textit{Hobbies}}: & Reading, Chess, Snowboarding, Surfing, Skateboarding, Horseback Riding, Hiking \\    
\end{tabular}



\end{document}
