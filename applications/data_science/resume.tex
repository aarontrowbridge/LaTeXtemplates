\documentclass[9pt]{extarticle}

\usepackage{extsizes}
\usepackage{hyperref}
\usepackage[margin=0.69in]{geometry}
\usepackage{paralist}

\setcounter{secnumdepth}{0}
\setlength{\parindent}{0pt}
\setlength{\parskip}{0pt}
\setlength{\pltopsep}{1pt}
\pagenumbering{gobble}

\newcommand{\myline}{\rule[\baselineskip]{\linewidth}{1pt}}

\begin{document}
\begin{center}
\Huge
\textbf{Aaron Trowbridge}\\

\normalsize
(610) 955-1580 $\cdot$ \href{mailto:aaron.j.trowbridge@gmail.com}{aaron.j.trowbridge@gmail.com} $\cdot$ \href{https://aarontrowbridge.github.io/}{aarontrowbridge.github.io} \\

\end{center}

% \Large{\textbf{Education}}

\section{Education}

\myline


\large\textbf{Syracuse University} 

\normalsize
\begin{compactitem}
\setlength\itemsep{0em}
\item B.S. in Physics, with distinction (3.6 GPA); B.S. in Mathematics (3.8 GPA) \hfill \small Sep 2015 -- Dec 2020 
\item One Semester as Graduate Physics Student and TA (4.0 GPA) [atypical situation due to COVID]
\end{compactitem}


\section{Experience}

\myline

\large\textbf{Private Tutor} \hfill \small Mar 2021 -- Present
\normalsize

\begin{compactitem}
\item Taught subjects including: physics, statistics, calculus, and programming in python
\end{compactitem}

\vspace{2.5pt}
\large\textbf{Teaching Assistant} \normalsize (Syracuse Physics Department)
\normalsize

\begin{compactitem}
\item One semester as graduate TA: introductory mechanics - under Prof. Walter Freeman \hfill \small Jan 2021 -- May 2021
\item Four semesters as undergrad TA: astronomy, mechanics, E $\&$ M, computational physics  \hfill \small Jan 2019 -- Dec 2020
\end{compactitem}

\vspace{2.5pt}
\large\textbf{Research Assistant} \normalsize (Syracuse Physics Department)
\normalsize

\begin{compactitem}
\item Lattice quantum gravity group, under Prof. Jack Laiho \hfill \small May 2020 -- May 2021 
\normalsize
\item \href{https://bplourde.expressions.syr.edu/}{\underline{Plourde Research Lab}} (superconducting quantum devices), under Prof. Britton Plourde \hfill \small May 2018 -- Dec 2020
\end{compactitem}




\section{Research $\&$ Projects}

\myline

\large\textbf{Personal Website and Blog}
\normalsize

\begin{compactitem}
\item Used the Hugo static website framework to develop my personal website and blog; learning HTML, CSS, Markdown, Go, and (just a little) JavaScript in the process. 
\item I have written \href{https://aarontrowbridge.github.io/posts/}{\underline{blog posts}} about interactive web-based plotting with Julia, quantum field theory, and quantitative finance, as well as other topics, and plan on writing more about data science and machine learning in the near future. 
\end{compactitem}

\vspace{2.5pt}
\large\textbf{Deep Generative Models}
\normalsize

\begin{compactitem}
\item Implemented generative adversarial networks for image generation from scratch in Julia using the Flux.jl machine learning library. Code can be found by clicking \href{https://github.com/aarontrowbridge/FluxGAN.jl}{\underline{here}} and a blog post \href{https://aarontrowbridge.github.io/posts/generative-adversarial-nets/}{\underline{here}}.
\item Implemented various variational autoencoder models in python using the PyTorch library. 
\item Read various papers about optimized generative techniques, including $\beta$-VAE, Wasserstein GAN/VAE, and Riemannian Manifold Hamiltonian Monte Carlo sampling. 
\item Currently researching other generative techniques, e.g. flow based models using gauge equivariant layers to sample configurations for lattice gauge theory (a technique for simulating quantum field theories).
\end{compactitem}


\vspace{2.5pt}
\large\textbf{Monte Carlo Methods for Lattice Quantum Gravity}
\normalsize

\begin{compactitem}
\item Worked with the lattice gravity group to develop, implement, and test a novel rejection free variant of the Metropolis algorithm.
\item MCMC methods, i.e. simulation of a stochastic process to estimate expectation values, are extremely useful in a wide range of subjects including machine learning, quantitative finance, and risk analysis.
\item A recorded talk I gave can be found on youtube by clicking \href{https://www.youtube.com/watch?v=_Ppx0e3aG-E&t=2s}{\underline{here}} and a github repo can be found \href{https://github.com/aarontrowbridge/Ising}{\underline{here}}.
\end{compactitem}

\vspace{2.5pt}
\large\textbf{Quantum Computer Simulation}
\normalsize

\begin{compactitem}
\item As final project for a quantum information theory course, I implemented a custom quantum gate programming language and virtual quantum processor, in Julia. A github repo can be found \href{https://github.com/aarontrowbridge/QuIPS}{\underline{here}}.
\item Quantum computation is becoming popular paradigm in various data analytical fields, where it can be used to efficiently calculate expectation values and assist with machine learning tasks.
\end{compactitem}

\section{Additional Information}
\myline

\normalsize

\begin{tabular}{ll}
\textbf{\textit{Programming Languages}}: & Julia, Python, C, C++,  Haskell, Bash, Git, HTML, CSS, Markdown, \LaTeX \\
\textbf{\textit{Operating Systems}}: & Linux (Arch, Manjaro i3), MacOS, Windows 10, Arduino, Visual Studio Code \\ 
\textbf{\textit{Hobbies}}: & Reading, Chess, Snowboarding, Surfing, Skateboarding, Horseback Riding, Hiking \\    
\end{tabular}



\end{document}
