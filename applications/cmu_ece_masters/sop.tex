\documentclass{article}

\title{\huge Statement of Purpose \\[1.5ex] \Large M.S. in ECE -- Carnegie Mellon University}
\author{Aaron Trowbridge}
\date{}

\pagenumbering{gobble}

\begin{document}

\maketitle

I have had a particularly nonlinear academic trajectory -- pivoting multiple times, dealing with failures, exploring various fields -- which has placed me in my current position: seemingly able to apply my unique skill set in an area I am deeply passionate about. 

Shortly after restarting my undergraduate career as a physics and math major (I was previously, unhappily a business major), I started my first undergraduate research job in a superconducting qubit lab.  This experience spurred my interest in quantum physics in general, and in particular---later on in my undergraduate research track---quantum field theory and quantum gravity.   

In parallel to my physics research, and not in the least decoupled from it, I had the pleasure of being an undergraduate TA for four separate physics courses: astronomy, mechanics, electricity \& magnetism, and computational physics.  Through teaching I acquired a much deeper understanding of fundamental physics, math, and computational concepts while also gaining an appreciation for the art of communicating technical ideas.

My final semester at my alma mater was, due to the pandemic, actually as a graduate student and teaching assistant.  I was given the responsibility of teaching four sections (two classes a week each) of introductory mechanics; additionally, I was taking two graduate courses.  It was a beautiful, demanding challenge, given the circumstances of the time and the fact that it was my first time leading any class.  Looking back, it was one of the happiest periods of my life, and I will forever remember the many amazing interactions I had with my students.

Now I find myself coming back to my first joys as an aspiring physicist: the intrigue that comes with experimentally interacting with matter at the extremes of size and temperature.  This past spring I walked into a professor's office here at CMU and asked if I could help with a research project applying trajectory optimization methods to superconducting quantum computers; today, pulses (generated by code I wrote) are directly controlling physical quantum hardware. The world we live in is a spooky and fascinatingly malleable place and I am elated at the potential opportunity I have before me to explore, experiment with, and shape that world. 


\end{document}