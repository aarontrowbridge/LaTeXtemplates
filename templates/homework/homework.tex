\documentclass[12pt]{article}
\usepackage[margin=1in]{geometry}
\usepackage[all]{xy}


\usepackage{amsmath,amsthm,amssymb,color,latexsym}
\usepackage{physics}
\usepackage{geometry}        
\geometry{letterpaper}    
\usepackage{graphicx}

\newtheorem{problem}{Problem}

\newenvironment{solution}[1][\it{Solution}]{\noindent\textbf{#1. } }{\hfill$\square$}


\begin{document}
\noindent Course Name \hfill Problem Set \#\\
Aaron Trowbridge. (MM/DD)

\hrulefill


\begin{problem}
  \hfill
  \begin{enumerate}
    \item[(a)] Write here the text of the first homework subproblem.
    \item[(b)] Write here the text of the second homework subproblem.
  \end{enumerate}
\end{problem}
\begin{solution}
  \begin{enumerate}
    \item[(a)] Write here the solution of the first homework subproblem.
    \item[(b)] Write here the solution of the second homework subproblem.
  \end{enumerate}
\end{solution} 

\begin{problem}
Write here the text of the second homework problem.
\end{problem}
\begin{solution}
	Write here the solution of the second homework problem.
\end{solution}

%%%%%%%%%%%%%%%%%%%%%%%%%%%%%%%%%%%%%%%%%%%%%%%%%%%%%%%%
%%%%%Continue with this pattern if there are more%%%%%%%
%%%%%%%%%%%%%%%%%homework problems%%%%%%%%%%%%%%%%%%%%%%
%%%%%%%%%%%%%%%%%%%%%%%%%%%%%%%%%%%%%%%%%%%%%%%%%%%%%%%%
 
\end{document}