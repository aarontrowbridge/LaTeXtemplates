\documentclass[9pt]{extarticle}

\usepackage{extsizes}
\usepackage{hyperref}
\usepackage[margin=0.69in]{geometry}
\usepackage{paralist}

\setcounter{secnumdepth}{0}
\setlength{\parindent}{0pt}
\setlength{\parskip}{0pt}
\setlength{\pltopsep}{1pt}
\pagenumbering{gobble}

\newcommand{\myline}{\rule[\baselineskip]{\linewidth}{1pt}}

\begin{document}
\begin{center}
\Huge
\textbf{Aaron Trowbridge}\\

\normalsize
(610) 955-1580 $\cdot$ \href{mailto:aaron.j.trowbridge@gmail.com}{aaron.j.trowbridge@gmail.com} $\cdot$ \href{https://aarontrowbridge.github.io/}{aarontrowbridge.github.io} \\

\end{center}

% \Large{\textbf{Education}}

\section{Education}

\myline


\large\textbf{Syracuse University} 

\normalsize
\begin{compactitem}
\setlength\itemsep{0em}
\item B.S. in Physics, with distinction (3.6 GPA); B.S. in Mathematics (3.8 GPA) \hfill \small Sep 2015 -- Dec 2020 
\item One Semester as Graduate Physics Student and TA (4.0 GPA), coursework: \\ Quantum Field Theory I; Lie groups, Lie algebras, and Representation Theory \hfill \small Jan 2021 -- May 2021
\end{compactitem}


\section{Experience}

\myline

\large\textbf{Private Tutor} \hfill \small Mar 2021 -- Present
\normalsize

\begin{compactitem}
\item Taught subjects including: physics, statistics, calculus, and programming in python
\end{compactitem}

\vspace{2.5pt}
\large\textbf{Teaching Assistant} \normalsize (Syracuse Physics Department)
\normalsize

\begin{compactitem}
\item One semester as graduate TA: introductory mechanics - under Prof. Walter Freeman \hfill \small Jan 2021 -- May 2021
\item Four semesters as undergrad TA: astronomy, mechanics, E $\&$ M, computational physics  \hfill \small Jan 2019 -- Dec 2020
\end{compactitem}

\vspace{2.5pt}
\large\textbf{Research Assistant} \normalsize (Syracuse Physics Department)
\normalsize

\begin{compactitem}
\item Lattice quantum gravity group, under Prof. Jack Laiho \hfill \small May 2020 -- May 2021 
\normalsize
\item \href{https://bplourde.expressions.syr.edu/}{\underline{Plourde Research Lab}} (superconducting quantum devices), under Prof. Britton Plourde \hfill \small May 2018 -- Dec 2020
\end{compactitem}




\section{Research $\&$ Projects}

\myline

\large\textbf{Personal Website and Blog}
\normalsize

\begin{compactitem}
\item Used the Hugo static website framework to develop my personal website and blog; learning HTML, CSS, Markdown, Go, and (just a little) JavaScript in the process. 
\item I have written \href{https://aarontrowbridge.github.io/posts/}{\underline{blog posts}} about interactive web-based plotting with Julia, quantum field theory, and quantitative finance, as well as other topics, and plan on writing more about data science and machine learning in the near future. 
\end{compactitem}

\vspace{2.5pt}
\large\textbf{Generative Adversarial Networks}
\normalsize

\begin{compactitem}
\item I am currently working on a project implementing various versions of the GAN unsupervised machine learning algorithm, using convolutional and fully connected neural networks to extract and utilize latent information of images to generate my own images in the same style. Code can be found by clicking \href{https://github.com/aarontrowbridge/FluxGAN.jl}{\underline{here}}.  
\end{compactitem}



\vspace{2.5pt}
\large\textbf{Gauge Theory and Topological Quantum Gravity}
\normalsize

\begin{compactitem}
\item As a final project for my QFT course I gave a talk on the theoretical aspects of quantizing gravity -- discussing the graviton propagator, gauge theory, topological field theory, and Chern-Simons gravity. Notes found \href{https://aarontrowbridge.github.io/notes/}{\underline{here}}.
\end{compactitem}


\vspace{2.5pt}
\large\textbf{Monte Carlo Methods for the Ising Model and Lattice Quantum Gravity}
\normalsize

\begin{compactitem}
\item Worked with the lattice gravity group to develop, implement, and test a novel rejection free variant of the Metropolis algorithm. 
\item A recorded talk I gave can be found on youtube by clicking \href{https://www.youtube.com/watch?v=_Ppx0e3aG-E&t=2s}{\underline{here}} and a github repo can be found \href{https://github.com/aarontrowbridge/Ising}{\underline{here}}.
\end{compactitem}

\vspace{2.5pt}
\large\textbf{Classical Simulation of a Quantum Computer}
\normalsize

\begin{compactitem}
\item As final project for a quantum information theory course, I implemented a custom quantum gate programming language and virtual quantum processor, in Julia. A github repo can be found \href{https://github.com/aarontrowbridge/QuIPS}{\underline{here}}.
\end{compactitem}


\vspace{2.5pt}
\large\textbf{Barnes-Hut Tree Algorithm for Gravitational $n$-body Simulation}
\normalsize

\begin{compactitem}
\item Implemented, in Julia, this $O(n \log n)$ tree-based approximation scheme. Code found \href{https://github.com/aarontrowbridge/Gravity}{\underline{here}}. 
\end{compactitem}


\vspace{2.5pt}
\large\textbf{Numerical Methods for Quantum Mechanics}
\normalsize

\begin{compactitem}
\item Under the supervision of Prof. Walter Freeman, I implemented and experimented with three methods for simulating quantum mechanical systems and solving the Schr\"odinger equation. Code found \href{https://github.com/aarontrowbridge/Quantum}{\underline{here}}. 
\end{compactitem}



\section{Additional Information}
\myline

\normalsize

\begin{tabular}{ll}
\textbf{\textit{Programming Languages}}: & Julia, Python, C, C++,  Haskell, Bash, Git, HTML, CSS, Markdown, \LaTeX \\
\textbf{\textit{Operating Systems}}: & Linux (Arch, Manjaro i3), MacOS, Windows 10, Arduino, Visual Studio Code \\ 
\textbf{\textit{Hobbies}}: & Reading, Chess, Snowboarding, Surfing, Skateboarding, Horseback Riding, Hiking \\    
\end{tabular}



\end{document}
